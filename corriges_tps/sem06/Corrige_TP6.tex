\documentclass[10pt]{article}

\usepackage{graphics}
\usepackage{dirtree}
\usepackage{paracol}
\usepackage{epigraph}
\usepackage{enumitem}
\usepackage{xcolor}

\setlength{\textwidth}{17cm}
\setlength{\oddsidemargin}{-1cm}
\setlength{\evensidemargin}{-1cm}
\setlength{\textheight}{26cm}
\setlength{\parindent}{0pt}
\setlength{\parskip}{0.3ex}

\input{preambule.tex}

\begin{document}
\thispagestyle{empty}

\definecolor{mygreen}{rgb}{0,0.6,0}
\definecolor{mygray}{rgb}{0.5,0.5,0.5}
\definecolor{mymauve}{rgb}{0.58,0,0.82}

\lstdefinestyle{customc}{
  belowcaptionskip=1\baselineskip,
  breaklines=true,
  frame=L,
  xleftmargin=\parindent,
  language=C,
  showstringspaces=false,
  basicstyle=\footnotesize\ttfamily,
  keywordstyle=\bfseries\color{green!40!black},
  commentstyle=\itshape\color{purple!40!black},
  identifierstyle=\color{blue},
  stringstyle=\color{orange},
}

\lstdefinestyle{none}{
  belowcaptionskip=1\baselineskip,
  breaklines=true,
  frame=L,
  xleftmargin=\parindent,
  language=bash,
  showstringspaces=false,
  basicstyle=\footnotesize\ttfamily,
  keywordstyle=\bfseries\color{black},
  commentstyle=\itshape\color{black},
  identifierstyle=\color{black},
  stringstyle=\color{black},
}

\lstset{tabsize=3, style=customc,literate=
  {á}{{\'a}}1 {é}{{\'e}}1 {í}{{\'i}}1 {ó}{{\'o}}1 {ú}{{\'u}}1
  {Á}{{\'A}}1 {É}{{\'E}}1 {Í}{{\'I}}1 {Ó}{{\'O}}1 {Ú}{{\'U}}1
  {à}{{\`a}}1 {è}{{\`e}}1 {ì}{{\`i}}1 {ò}{{\`o}}1 {ù}{{\`u}}1
  {À}{{\`A}}1 {È}{{\'E}}1 {Ì}{{\`I}}1 {Ò}{{\`O}}1 {Ù}{{\`U}}1
  {ä}{{\"a}}1 {ë}{{\"e}}1 {ï}{{\"i}}1 {ö}{{\"o}}1 {ü}{{\"u}}1
  {Ä}{{\"A}}1 {Ë}{{\"E}}1 {Ï}{{\"I}}1 {Ö}{{\"O}}1 {Ü}{{\"U}}1
  {â}{{\^a}}1 {ê}{{\^e}}1 {î}{{\^i}}1 {ô}{{\^o}}1 {û}{{\^u}}1
  {Â}{{\^A}}1 {Ê}{{\^E}}1 {Î}{{\^I}}1 {Ô}{{\^O}}1 {Û}{{\^U}}1
  {œ}{{\oe}}1 {Œ}{{\OE}}1 {æ}{{\ae}}1 {Æ}{{\AE}}1 {ß}{{\ss}}1
  {ű}{{\H{u}}}1 {Ű}{{\H{U}}}1 {ő}{{\H{o}}}1 {Ő}{{\H{O}}}1
  {ç}{{\c c}}1 {Ç}{{\c C}}1 {ø}{{\o}}1 {å}{{\r a}}1 {Å}{{\r A}}1
  {€}{{\euro}}1 {£}{{\pounds}}1 {«}{{\guillemotleft}}1
  {»}{{\guillemotright}}1 {ñ}{{\~n}}1 {Ñ}{{\~N}}1 {¿}{{?`}}1
}

\EnteteUE

\begin{center}
  {\large {\bf Corrigé TP6}}
\end{center}

\vspace*{0.5cm}

\begin{enumerate}[label=\textbf{[\alph*]}]
  \setlength\itemsep{1em}

\item Le type qui définira notre ensemble est \texttt{chaines}. Il
  s'agit d'une structure contenant le cardinal de l'ensemble ainsi que
  les chaînes contenues dans un tableau. Le tableau est déclaré
  statiquement et l'on voit d'après les constantes
  \texttt{NB\_MAX\_CHAINES} et \texttt{TAILLE\_MAX\_CHAINES} que l'on
  pourra stocker au maximum 128 chaînes et que chacune fera au maximum
  128 caractères.

\item Pour compiler le programme, on tape simplement
  \texttt{make}. L'exécution de \texttt{test\_chaines}, l'exécutable
  obtenu, ne donne effectivement pas le résultat attendu.

\begin{verbatim}
$ ./test_chaines

Copie successive des arguments dans l'ensemble :
 bidule - machin - chose -
Ensemble de chaines dans lequel la recherche est faite :
Recherche de truc dans cet ensemble :
 -- Chaine absente de l'ensemble
\end{verbatim}

\item Voici une implémentation possible.

  \lstinputlisting{listing/chaines.c}

\item La commande \texttt{wc} permet de compter les mots d'un fichier
  ou d'un flux de caractère. Les options se déclinent ainsi :

  \begin{description}
  \item[-w] Compte les mots
  \item[-c] Compte les caractères
  \item[-l] Compte les lignes
  \end{description}

\item Voilà le résultat de la commande :

\begin{verbatim}
$ grep chateau Candide_chapitre1.txt | wc -l
10
\end{verbatim}

  On sait que la commande \texttt{grep} affiche toutes les lignes d'un
  fichier contenant le mot passer en paramètre. Le pipe suivi de la
  commande \texttt{wc -l} permet de compter les lignes issues du
  résultat du \texttt{grep}. On a donc le nombre d'occurence du mot
  ``chateau'' dans le texte.

\item Avec la commande suivante :

\begin{verbatim}
  $ grep Cunegonde Candide_chapitre1.txt | grep Candide | wc -l
\end{verbatim}

  On obtient le nombre de lignes contenant à la fois ``Cunégonde'' et
  ``Candide''. Une seule ligne répond à la condition.

\item Le fichier \texttt{client.h} contient la définition des types
  représentant un client, \texttt{client} et un ensemble de clients,
  \texttt{ensemble\_clients}. On pourrait imaginer avoir les valeurs
  suivantes :

\begin{verbatim}
typedef struct {
        char nom[LGNOM_MAX];     -->  "Dupond"
        int compte;              -->  12
} client;

typedef struct {
        client T[NB_MAX];        -->  { client1, client2, client3 }
        int nb;                  -->  3
} ensemble_clients;

\end{verbatim}

\item Pour compiler le fichier objet \texttt{clients\_in.o}, on
  utilise la commande suivante :

\begin{verbatim}
  gcc -Wall -Wextra -Werror -c clients_in.c
\end{verbatim}

  NB: Toutes les options -W sont facultatives, mais fournissent une
  aide précieuse.

  On obtient le message d'erreur :

\begin{verbatim}
clients_in.c:7:23: erreur : unknown type name « ensemble_clients »
 void lire_les_clients(ensemble_clients *e, char *nom_fich) {
                       ^~~~~~~~~~~~~~~~

\end{verbatim}

  Cela signifie que le type \texttt{ensemble\_clients} n'est pas
  défini. Or, sa définition se trouve dans le fichier
  \texttt{clients.h}. Cela nous met sur la voie d'un fichier header
  manquant. Après investigation, on s'aperçoit qu'il manque le
  include~: \texttt{\#include ``clients\_in.h''} qui lui-même contient
  le include~: \texttt{\#include ``clients.h''}

\item Les cibles des Makefile ont toujours la syntaxe suivante :

\begin{verbatim}
CIBLE: DEPENDENCE1 DEPENDENCE2 ...
    REGLE_POUR_CREER_LA_CIBLE
^~~~
  |
Ceci est une TABULATION et certainement pas des espaces
\end{verbatim}

  Pour créer le fichier objet \texttt{fich.o}, la syntaxe de la cible
  Makefile sera de cette forme :

\begin{verbatim}
fich.o: fich.c fich.h autre_header.h
    gcc -Wall -Wextra -Werror -c fich.c
^~~~                          ^~
  |                            |
Toujours une TAB       L'option -c fait la compilation partielle
\end{verbatim}

\item L'exécutable \texttt{gestion} a besoin de tous les fichiers
  objet générés précédemment. Soit :

  \begin{itemize}
  \item gestion.o
  \item clients\_in.o
  \item clients\_out.o
  \item clients.o
  \item operations.o
  \end{itemize}

  L'étape de création de l'exécutable s'appelle
  \textbf{l'édition de lien}. À ce moment, on a fini de compiler tous
  les fichiers source en fichiers objet. On n'a donc plus besoin des
  \texttt{.c} ni des \texttt{.h}. Il ne reste plus qu'à faire le lien
  entre les fonctions compilées et leurs appels.

\item On se souvient que la commande \texttt{touch} permet de créer un
  fichier vide ou de modifier la date de dernière modification d'un
  fichier existant. On utilise donc la commande :

\begin{verbatim}
  touch *.c *.h
\end{verbatim}

\item Graphe de dépendences :

\begin{verbatim}
clients.c ------- \
                   ->   clients.o         \
clients.h ------- /                        |
   ^                                       |
   |                                       |
   |                                       |
clients_in.c ---- \                        |
   |               \                       |
   +-------------   ->   clients_in.o      |
   |               /                       |
clients_in.h ---- /                        |
   | ^                                     |
   | |                                     |
   | |                                     |
clients_out.c --- \
   | |             \
   +-------------   ->   clients_out.o
   | |             /
clients_out.h --- /
   | | ^
   | | |
   | | |
operations.c ---- \
   | | |           \
   +-------------   ->   operations.o
   | | |           /
operations.h ---- /
   | | | ^
   | | | |
   | | | |
   | | | +------- \
   | | +---------  \
   | +-----------   \
   +-------------    -> gestion.o
                    /
                   /
gestion.c ------- /

\end{verbatim}

\end{enumerate}

\end{document}
