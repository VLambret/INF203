\documentclass[10pt]{article}

\usepackage{graphics}
\usepackage{dirtree}
\usepackage{paracol}
\usepackage{epigraph}
\usepackage{enumitem}
\usepackage{xcolor}
\usepackage{tikz}

\setlength{\textwidth}{17cm}
\setlength{\oddsidemargin}{-1cm}
\setlength{\evensidemargin}{-1cm}
\setlength{\textheight}{26cm}
\setlength{\parindent}{0pt}
\setlength{\parskip}{0.3ex}

\usepackage{fancyhdr}
\usepackage{epsfig}
\usepackage[utf8]{inputenc}
\usepackage[french]{babel}
\usepackage[T1]{fontenc}
%\usepackage{verbatim}
\usepackage{graphics}
\usepackage{amsmath,amsfonts,amssymb}
\usepackage{listings}
\usepackage{thmbox}
\usepackage{comment}

\oddsidemargin=0cm
\evensidemargin=0cm
\textwidth=17cm
\textheight=25cm
\topmargin=0mm
\hoffset=-6mm
\voffset=-25mm
%\headsep=30pt

\newcounter{numpartie}
\newcounter{exo}
\newcommand{\partie}[2]
{
        \refstepcounter{numpartie}
        \setcounter{exo}{0}
        \vspace{3mm}

        %\noindent{\large\bf Partie \Alph{numpartie}~-~#1}\hfill
        \noindent{\large\bf Partie \Alph{numpartie}~ ~#1}\hfill
        %[\;{\it bar\^eme indicatif : #2}\;]\\
        \\

}

\newenvironment{exercice}{\refstepcounter{exo} \vspace*{1em}\begin{thmbox}[S]{{\bf Exercice \arabic{exo}~:}}}{\end{thmbox}}

\newenvironment{solution}{\vspace*{1em}\begin{thmbox}[S]{{\bf Solution~:}}}{\end{thmbox}}

\newcommand{\IGNORE}[1]{}

\def\EnteteUE{
\noindent{\bf Université Grenoble Alpes} \hfill {\bf DLST}

\vspace{-8pt}

\noindent\hrulefill

\vspace{-2pt}

\noindent{\bf UE INF203} \hfill {\bf Année 2017-18}

\vspace*{.5cm}
}

\newcommand{\PiedDePageUE}[1]{
\fancypagestyle{monstyle}
{
\rhead{}
\lfoot{{\bf\small INF203 - 2016/2017}}
\cfoot{\thepage}
\rfoot{{\it #1}}
\renewcommand{\footrulewidth}{0.5pt}
\renewcommand{\headrulewidth}{0.0pt}
}
\pagestyle{monstyle}
}


\newcounter{question}
\newenvironment{question}[1][]{\refstepcounter{question}
   \textbf{[\bf \alph{question}] {\begin{emph}{#1}\end{emph}}}}{~$\blacksquare$~\\}

\newcommand{\unix}[1]{\hspace*{2cm}{\bf \tt #1}}
\newcommand{\fich}[1]{{\bf \em #1}}
\newcommand{\ecran}[1]{{\tt #1}}
\newcommand{\C}[1]{{\tt #1}}
\newcommand{\cmd}[1]{{\bf #1}}
\newcommand{\home}{\~{}}

\newenvironment{exocomp}{\vspace*{1em}\begin{thmbox}[M]{Exercice complémentaire :~}}{\end{thmbox}}


\begin{document}
\thispagestyle{empty}

\definecolor{mygreen}{rgb}{0,0.6,0}
\definecolor{mygray}{rgb}{0.5,0.5,0.5}
\definecolor{mymauve}{rgb}{0.58,0,0.82}

\lstdefinestyle{customc}{
  belowcaptionskip=1\baselineskip,
  breaklines=true,
  frame=L,
  xleftmargin=\parindent,
  language=C,
  showstringspaces=false,
  basicstyle=\footnotesize\ttfamily,
  keywordstyle=\bfseries\color{green!40!black},
  commentstyle=\itshape\color{purple!40!black},
  identifierstyle=\color{blue},
  stringstyle=\color{orange},
}

\lstdefinestyle{none}{
  belowcaptionskip=1\baselineskip,
  breaklines=true,
  frame=L,
  xleftmargin=\parindent,
  language=bash,
  showstringspaces=false,
  basicstyle=\footnotesize\ttfamily,
  keywordstyle=\bfseries\color{black},
  commentstyle=\itshape\color{black},
  identifierstyle=\color{black},
  stringstyle=\color{black},
}

\lstset{tabsize=3, style=customc,literate=
  {á}{{\'a}}1 {é}{{\'e}}1 {í}{{\'i}}1 {ó}{{\'o}}1 {ú}{{\'u}}1
  {Á}{{\'A}}1 {É}{{\'E}}1 {Í}{{\'I}}1 {Ó}{{\'O}}1 {Ú}{{\'U}}1
  {à}{{\`a}}1 {è}{{\`e}}1 {ì}{{\`i}}1 {ò}{{\`o}}1 {ù}{{\`u}}1
  {À}{{\`A}}1 {È}{{\'E}}1 {Ì}{{\`I}}1 {Ò}{{\`O}}1 {Ù}{{\`U}}1
  {ä}{{\"a}}1 {ë}{{\"e}}1 {ï}{{\"i}}1 {ö}{{\"o}}1 {ü}{{\"u}}1
  {Ä}{{\"A}}1 {Ë}{{\"E}}1 {Ï}{{\"I}}1 {Ö}{{\"O}}1 {Ü}{{\"U}}1
  {â}{{\^a}}1 {ê}{{\^e}}1 {î}{{\^i}}1 {ô}{{\^o}}1 {û}{{\^u}}1
  {Â}{{\^A}}1 {Ê}{{\^E}}1 {Î}{{\^I}}1 {Ô}{{\^O}}1 {Û}{{\^U}}1
  {œ}{{\oe}}1 {Œ}{{\OE}}1 {æ}{{\ae}}1 {Æ}{{\AE}}1 {ß}{{\ss}}1
  {ű}{{\H{u}}}1 {Ű}{{\H{U}}}1 {ő}{{\H{o}}}1 {Ő}{{\H{O}}}1
  {ç}{{\c c}}1 {Ç}{{\c C}}1 {ø}{{\o}}1 {å}{{\r a}}1 {Å}{{\r A}}1
  {€}{{\euro}}1 {£}{{\pounds}}1 {«}{{\guillemotleft}}1
  {»}{{\guillemotright}}1 {ñ}{{\~n}}1 {Ñ}{{\~N}}1 {¿}{{?`}}1
}

\EnteteUE

\begin{center}
  {\large {\bf Corrigé TP6}}
\end{center}

\vspace*{0.5cm}

\begin{enumerate}[label=\textbf{[\alph*]}]
  \setlength\itemsep{1em}

\item Le type qui définira notre ensemble est \texttt{chaines}. Il
  s'agit d'une structure contenant le cardinal de l'ensemble ainsi que
  les chaînes contenues dans un tableau. Le tableau est déclaré
  statiquement et l'on voit d'après les constantes
  \texttt{NB\_MAX\_CHAINES} et \texttt{TAILLE\_MAX\_CHAINES} que l'on
  pourra stocker au maximum 128 chaînes et que chacune fera au maximum
  128 caractères.

\item Pour compiler le programme, on tape simplement
  \texttt{make}. L'exécution de \texttt{test\_chaines}, l'exécutable
  obtenu, ne donne effectivement pas le résultat attendu.

\begin{verbatim}
$ ./test_chaines

Copie successive des arguments dans l'ensemble :
 bidule - machin - chose -
Ensemble de chaines dans lequel la recherche est faite :
Recherche de truc dans cet ensemble :
 -- Chaine absente de l'ensemble
\end{verbatim}

\item Voici une implémentation possible.

  \lstinputlisting{listing/chaines.c}

\item La commande \texttt{wc} permet de compter les mots d'un fichier
  ou d'un flux de caractère. Les options se déclinent ainsi :

  \begin{description}
  \item[-w] Compte les mots
  \item[-c] Compte les caractères
  \item[-l] Compte les lignes
  \end{description}

\item Voilà le résultat de la commande :

\begin{verbatim}
$ grep chateau Candide_chapitre1.txt | wc -l
10
\end{verbatim}

  On sait que la commande \texttt{grep} affiche toutes les lignes d'un
  fichier contenant le mot passer en paramètre. Le pipe suivi de la
  commande \texttt{wc -l} permet de compter les lignes issues du
  résultat du \texttt{grep}. On a donc le nombre d'occurence du mot
  ``chateau'' dans le texte.

\item Avec la commande suivante :

\begin{verbatim}
  $ grep Cunegonde Candide_chapitre1.txt | grep Candide | wc -l
\end{verbatim}

  On obtient le nombre de lignes contenant à la fois ``Cunégonde'' et
  ``Candide''. Une seule ligne répond à la condition.

\item Le fichier \texttt{client.h} contient la définition des types
  représentant un client, \texttt{client} et un ensemble de clients,
  \texttt{ensemble\_clients}. On pourrait imaginer avoir les valeurs
  suivantes :

\begin{verbatim}
{
    "T" : [
        {
            "nom" : [ "D", "u", "p", "o", "n", "d", "\0" ],
            "compte" : 100
        },
        {
            "nom" : [ "M", "a", "r", "t", "i", "n", "\0" ],
            "compte" : 40
        },
        {
            "nom" : [ "J", "o", "h", "n", "s", "o", "n", "\0" ],
            "compte" : 250
        }
    ],
    "nb" : 3
}
\end{verbatim}

\item Pour compiler le fichier objet \texttt{clients\_in.o}, on
  utilise la commande suivante :

\begin{verbatim}
  gcc -Wall -Wextra -Werror -c clients_in.c
\end{verbatim}

  NB: Toutes les options -W sont facultatives, mais fournissent une
  aide précieuse.

  On obtient le message d'erreur :

\begin{verbatim}
clients_in.c:7:23: erreur_: unknown type name «_ensemble_clients_»
 void lire_les_clients(ensemble_clients *e, char *nom_fich) {
                       ^~~~~~~~~~~~~~~~
\end{verbatim}

  Cela signifie que le type \texttt{ensemble\_clients} n'est pas
  défini. Or, sa définition se trouve dans le fichier
  \texttt{clients.h}. Cela nous met sur la voie d'un fichier header
  manquant. Après investigation, on s'aperçoit qu'il manque le
  include~: \texttt{\#include ``clients\_in.h''} qui lui-même contient
  le include~: \texttt{\#include ``clients.h''}

\item Les cibles des Makefile ont toujours la syntaxe suivante :

\begin{verbatim}
CIBLE: DEPENDENCE1 DEPENDENCE2 ...
    REGLE_POUR_CREER_LA_CIBLE
^~~~
  |
Ceci est une TABULATION et certainement pas des espaces
\end{verbatim}

  Pour créer le fichier objet \texttt{fich.o}, la syntaxe de la cible
  Makefile sera de cette forme :

\begin{verbatim}
fich.o: fich.c fich.h autre_header.h
    gcc -Wall -Wextra -Werror -c fich.c
^~~~                          ^~
  |                            |
Toujours une TAB       L'option -c fait la compilation partielle
\end{verbatim}

\item L'exécutable \texttt{gestion} a besoin de tous les fichiers
  objet générés précédemment. Soit :

  \begin{itemize}
  \item gestion.o
  \item clients\_in.o
  \item clients\_out.o
  \item clients.o
  \item operations.o
  \end{itemize}

  L'étape de création de l'exécutable s'appelle
  \textbf{l'édition de lien}. À ce moment, on a fini de compiler tous
  les fichiers source en fichiers objet. On n'a donc plus besoin des
  \texttt{.c} ni des \texttt{.h}. Il ne reste plus qu'à faire le lien
  entre les fonctions compilées et leurs appels.

\item On se souvient que la commande \texttt{touch} permet de créer un
  fichier vide ou de modifier la date de dernière modification d'un
  fichier existant. On utilise donc la commande :

\begin{verbatim}
  touch *.c *.h
\end{verbatim}

\newpage
\item Graphe de dépendences :

  % Graphic for TeX using PGF
% Title: /home/corentin/Diagramme1.dia
% Creator: Dia v0.97.3
% CreationDate: Sun Apr 30 19:05:16 2017
% For: corentin
% \usepackage{tikz}
% The following commands are not supported in PSTricks at present
% We define them conditionally, so when they are implemented,
% this pgf file will use them.
\ifx\du\undefined
  \newlength{\du}
\fi
\setlength{\du}{15\unitlength}
\begin{tikzpicture}
\pgftransformxscale{.650000}
\pgftransformyscale{-0.650000}
\definecolor{dialinecolor}{rgb}{0.000000, 0.000000, 0.000000}
\pgfsetstrokecolor{dialinecolor}
\definecolor{dialinecolor}{rgb}{1.000000, 1.000000, 1.000000}
\pgfsetfillcolor{dialinecolor}
\definecolor{dialinecolor}{rgb}{1.000000, 1.000000, 1.000000}
\pgfsetfillcolor{dialinecolor}
\pgfpathellipse{\pgfpoint{16.356322\du}{2.517044\du}}{\pgfpoint{2.443678\du}{0\du}}{\pgfpoint{0\du}{1.196839\du}}
\pgfusepath{fill}
\pgfsetlinewidth{0.100000\du}
\pgfsetdash{}{0pt}
\pgfsetdash{}{0pt}
\pgfsetmiterjoin
\definecolor{dialinecolor}{rgb}{0.000000, 0.000000, 0.000000}
\pgfsetstrokecolor{dialinecolor}
\pgfpathellipse{\pgfpoint{16.356322\du}{2.517044\du}}{\pgfpoint{2.443678\du}{0\du}}{\pgfpoint{0\du}{1.196839\du}}
\pgfusepath{stroke}
% setfont left to latex
\definecolor{dialinecolor}{rgb}{0.000000, 0.000000, 0.000000}
\pgfsetstrokecolor{dialinecolor}
\node at (16.356322\du,2.712044\du){clients.c};
\definecolor{dialinecolor}{rgb}{1.000000, 1.000000, 1.000000}
\pgfsetfillcolor{dialinecolor}
\pgfpathellipse{\pgfpoint{21.350000\du}{2.589582\du}}{\pgfpoint{2.338755\du}{0\du}}{\pgfpoint{0\du}{1.169377\du}}
\pgfusepath{fill}
\pgfsetlinewidth{0.100000\du}
\pgfsetdash{}{0pt}
\pgfsetdash{}{0pt}
\pgfsetmiterjoin
\definecolor{dialinecolor}{rgb}{0.000000, 0.000000, 1.000000}
\pgfsetstrokecolor{dialinecolor}
\pgfpathellipse{\pgfpoint{21.350000\du}{2.589582\du}}{\pgfpoint{2.338755\du}{0\du}}{\pgfpoint{0\du}{1.169377\du}}
\pgfusepath{stroke}
% setfont left to latex
\definecolor{dialinecolor}{rgb}{0.000000, 0.000000, 0.000000}
\pgfsetstrokecolor{dialinecolor}
\node at (21.350000\du,2.784582\du){clients.h};
\definecolor{dialinecolor}{rgb}{1.000000, 1.000000, 1.000000}
\pgfsetfillcolor{dialinecolor}
\pgfpathellipse{\pgfpoint{34.169624\du}{2.504461\du}}{\pgfpoint{2.548476\du}{0\du}}{\pgfpoint{0\du}{1.484256\du}}
\pgfusepath{fill}
\pgfsetlinewidth{0.100000\du}
\pgfsetdash{}{0pt}
\pgfsetdash{}{0pt}
\pgfsetmiterjoin
\definecolor{dialinecolor}{rgb}{0.000000, 0.000000, 0.000000}
\pgfsetstrokecolor{dialinecolor}
\pgfpathellipse{\pgfpoint{34.169624\du}{2.504461\du}}{\pgfpoint{2.548476\du}{0\du}}{\pgfpoint{0\du}{1.484256\du}}
\pgfusepath{stroke}
% setfont left to latex
\definecolor{dialinecolor}{rgb}{0.000000, 0.000000, 0.000000}
\pgfsetstrokecolor{dialinecolor}
\node at (34.169624\du,2.699461\du){clients\_in.c};
\definecolor{dialinecolor}{rgb}{1.000000, 1.000000, 1.000000}
\pgfsetfillcolor{dialinecolor}
\pgfpathellipse{\pgfpoint{39.650000\du}{2.565137\du}}{\pgfpoint{2.689865\du}{0\du}}{\pgfpoint{0\du}{1.344933\du}}
\pgfusepath{fill}
\pgfsetlinewidth{0.100000\du}
\pgfsetdash{}{0pt}
\pgfsetdash{}{0pt}
\pgfsetmiterjoin
\definecolor{dialinecolor}{rgb}{0.000000, 0.000000, 0.000000}
\pgfsetstrokecolor{dialinecolor}
\pgfpathellipse{\pgfpoint{39.650000\du}{2.565137\du}}{\pgfpoint{2.689865\du}{0\du}}{\pgfpoint{0\du}{1.344933\du}}
\pgfusepath{stroke}
% setfont left to latex
\definecolor{dialinecolor}{rgb}{0.000000, 0.000000, 0.000000}
\pgfsetstrokecolor{dialinecolor}
\node at (39.650000\du,2.760137\du){clients\_in.h};
\definecolor{dialinecolor}{rgb}{1.000000, 1.000000, 1.000000}
\pgfsetfillcolor{dialinecolor}
\pgfpathellipse{\pgfpoint{3.800000\du}{2.450000\du}}{\pgfpoint{2.859591\du}{0\du}}{\pgfpoint{0\du}{1.429795\du}}
\pgfusepath{fill}
\pgfsetlinewidth{0.100000\du}
\pgfsetdash{}{0pt}
\pgfsetdash{}{0pt}
\pgfsetmiterjoin
\definecolor{dialinecolor}{rgb}{0.000000, 0.000000, 0.000000}
\pgfsetstrokecolor{dialinecolor}
\pgfpathellipse{\pgfpoint{3.800000\du}{2.450000\du}}{\pgfpoint{2.859591\du}{0\du}}{\pgfpoint{0\du}{1.429795\du}}
\pgfusepath{stroke}
% setfont left to latex
\definecolor{dialinecolor}{rgb}{0.000000, 0.000000, 0.000000}
\pgfsetstrokecolor{dialinecolor}
\node at (3.800000\du,2.645000\du){clients\_out.c};
\definecolor{dialinecolor}{rgb}{1.000000, 1.000000, 1.000000}
\pgfsetfillcolor{dialinecolor}
\pgfpathellipse{\pgfpoint{9.800000\du}{2.511249\du}}{\pgfpoint{2.882088\du}{0\du}}{\pgfpoint{0\du}{1.441044\du}}
\pgfusepath{fill}
\pgfsetlinewidth{0.100000\du}
\pgfsetdash{}{0pt}
\pgfsetdash{}{0pt}
\pgfsetmiterjoin
\definecolor{dialinecolor}{rgb}{0.000000, 0.000000, 0.000000}
\pgfsetstrokecolor{dialinecolor}
\pgfpathellipse{\pgfpoint{9.800000\du}{2.511249\du}}{\pgfpoint{2.882088\du}{0\du}}{\pgfpoint{0\du}{1.441044\du}}
\pgfusepath{stroke}
% setfont left to latex
\definecolor{dialinecolor}{rgb}{0.000000, 0.000000, 0.000000}
\pgfsetstrokecolor{dialinecolor}
\node at (9.800000\du,2.706249\du){clients\_out.h};
\definecolor{dialinecolor}{rgb}{1.000000, 1.000000, 1.000000}
\pgfsetfillcolor{dialinecolor}
\pgfpathellipse{\pgfpoint{46.650000\du}{2.450000\du}}{\pgfpoint{2.838575\du}{0\du}}{\pgfpoint{0\du}{1.419287\du}}
\pgfusepath{fill}
\pgfsetlinewidth{0.100000\du}
\pgfsetdash{}{0pt}
\pgfsetdash{}{0pt}
\pgfsetmiterjoin
\definecolor{dialinecolor}{rgb}{0.000000, 0.000000, 0.000000}
\pgfsetstrokecolor{dialinecolor}
\pgfpathellipse{\pgfpoint{46.650000\du}{2.450000\du}}{\pgfpoint{2.838575\du}{0\du}}{\pgfpoint{0\du}{1.419287\du}}
\pgfusepath{stroke}
% setfont left to latex
\definecolor{dialinecolor}{rgb}{0.000000, 0.000000, 0.000000}
\pgfsetstrokecolor{dialinecolor}
\node at (46.650000\du,2.645000\du){operations.c};
\definecolor{dialinecolor}{rgb}{1.000000, 1.000000, 1.000000}
\pgfsetfillcolor{dialinecolor}
\pgfpathellipse{\pgfpoint{52.550000\du}{2.492459\du}}{\pgfpoint{2.723493\du}{0\du}}{\pgfpoint{0\du}{1.361747\du}}
\pgfusepath{fill}
\pgfsetlinewidth{0.100000\du}
\pgfsetdash{}{0pt}
\pgfsetdash{}{0pt}
\pgfsetmiterjoin
\definecolor{dialinecolor}{rgb}{0.000000, 0.000000, 0.000000}
\pgfsetstrokecolor{dialinecolor}
\pgfpathellipse{\pgfpoint{52.550000\du}{2.492459\du}}{\pgfpoint{2.723493\du}{0\du}}{\pgfpoint{0\du}{1.361747\du}}
\pgfusepath{stroke}
% setfont left to latex
\definecolor{dialinecolor}{rgb}{0.000000, 0.000000, 0.000000}
\pgfsetstrokecolor{dialinecolor}
\node at (52.550000\du,2.687459\du){operation.h};
\definecolor{dialinecolor}{rgb}{1.000000, 1.000000, 1.000000}
\pgfsetfillcolor{dialinecolor}
\pgfpathellipse{\pgfpoint{27.650000\du}{2.190538\du}}{\pgfpoint{2.419651\du}{0\du}}{\pgfpoint{0\du}{1.209826\du}}
\pgfusepath{fill}
\pgfsetlinewidth{0.100000\du}
\pgfsetdash{}{0pt}
\pgfsetdash{}{0pt}
\pgfsetmiterjoin
\definecolor{dialinecolor}{rgb}{0.000000, 0.000000, 0.000000}
\pgfsetstrokecolor{dialinecolor}
\pgfpathellipse{\pgfpoint{27.650000\du}{2.190538\du}}{\pgfpoint{2.419651\du}{0\du}}{\pgfpoint{0\du}{1.209826\du}}
\pgfusepath{stroke}
% setfont left to latex
\definecolor{dialinecolor}{rgb}{0.000000, 0.000000, 0.000000}
\pgfsetstrokecolor{dialinecolor}
\node at (27.650000\du,2.385538\du){gestion.c};
\definecolor{dialinecolor}{rgb}{1.000000, 1.000000, 1.000000}
\pgfsetfillcolor{dialinecolor}
\pgfpathellipse{\pgfpoint{19.050000\du}{13.300000\du}}{\pgfpoint{2.334013\du}{0\du}}{\pgfpoint{0\du}{1.167007\du}}
\pgfusepath{fill}
\pgfsetlinewidth{0.100000\du}
\pgfsetdash{}{0pt}
\pgfsetdash{}{0pt}
\pgfsetmiterjoin
\definecolor{dialinecolor}{rgb}{0.000000, 0.000000, 0.000000}
\pgfsetstrokecolor{dialinecolor}
\pgfpathellipse{\pgfpoint{19.050000\du}{13.300000\du}}{\pgfpoint{2.334013\du}{0\du}}{\pgfpoint{0\du}{1.167007\du}}
\pgfusepath{stroke}
% setfont left to latex
\definecolor{dialinecolor}{rgb}{0.000000, 0.000000, 0.000000}
\pgfsetstrokecolor{dialinecolor}
\node at (19.050000\du,13.495000\du){clients.o};
\definecolor{dialinecolor}{rgb}{1.000000, 1.000000, 1.000000}
\pgfsetfillcolor{dialinecolor}
\pgfpathellipse{\pgfpoint{5.800000\du}{13.750000\du}}{\pgfpoint{2.738729\du}{0\du}}{\pgfpoint{0\du}{1.369365\du}}
\pgfusepath{fill}
\pgfsetlinewidth{0.100000\du}
\pgfsetdash{}{0pt}
\pgfsetdash{}{0pt}
\pgfsetmiterjoin
\definecolor{dialinecolor}{rgb}{0.000000, 0.000000, 0.000000}
\pgfsetstrokecolor{dialinecolor}
\pgfpathellipse{\pgfpoint{5.800000\du}{13.750000\du}}{\pgfpoint{2.738729\du}{0\du}}{\pgfpoint{0\du}{1.369365\du}}
\pgfusepath{stroke}
% setfont left to latex
\definecolor{dialinecolor}{rgb}{0.000000, 0.000000, 0.000000}
\pgfsetstrokecolor{dialinecolor}
\node at (5.800000\du,13.945000\du){client\_out.o};
\definecolor{dialinecolor}{rgb}{1.000000, 1.000000, 1.000000}
\pgfsetfillcolor{dialinecolor}
\pgfpathellipse{\pgfpoint{37.900000\du}{13.650000\du}}{\pgfpoint{2.684544\du}{0\du}}{\pgfpoint{0\du}{1.342272\du}}
\pgfusepath{fill}
\pgfsetlinewidth{0.100000\du}
\pgfsetdash{}{0pt}
\pgfsetdash{}{0pt}
\pgfsetmiterjoin
\definecolor{dialinecolor}{rgb}{0.000000, 0.000000, 0.000000}
\pgfsetstrokecolor{dialinecolor}
\pgfpathellipse{\pgfpoint{37.900000\du}{13.650000\du}}{\pgfpoint{2.684544\du}{0\du}}{\pgfpoint{0\du}{1.342272\du}}
\pgfusepath{stroke}
% setfont left to latex
\definecolor{dialinecolor}{rgb}{0.000000, 0.000000, 0.000000}
\pgfsetstrokecolor{dialinecolor}
\node at (37.900000\du,13.845000\du){clients\_in.o};
\definecolor{dialinecolor}{rgb}{1.000000, 1.000000, 1.000000}
\pgfsetfillcolor{dialinecolor}
\pgfpathellipse{\pgfpoint{49.800000\du}{13.700000\du}}{\pgfpoint{2.855775\du}{0\du}}{\pgfpoint{0\du}{1.427887\du}}
\pgfusepath{fill}
\pgfsetlinewidth{0.100000\du}
\pgfsetdash{}{0pt}
\pgfsetdash{}{0pt}
\pgfsetmiterjoin
\definecolor{dialinecolor}{rgb}{0.000000, 0.000000, 0.000000}
\pgfsetstrokecolor{dialinecolor}
\pgfpathellipse{\pgfpoint{49.800000\du}{13.700000\du}}{\pgfpoint{2.855775\du}{0\du}}{\pgfpoint{0\du}{1.427887\du}}
\pgfusepath{stroke}
% setfont left to latex
\definecolor{dialinecolor}{rgb}{0.000000, 0.000000, 0.000000}
\pgfsetstrokecolor{dialinecolor}
\node at (49.800000\du,13.895000\du){operations.o};
\definecolor{dialinecolor}{rgb}{1.000000, 1.000000, 1.000000}
\pgfsetfillcolor{dialinecolor}
\pgfpathellipse{\pgfpoint{27.946185\du}{13.506620\du}}{\pgfpoint{2.521926\du}{0\du}}{\pgfpoint{0\du}{1.164960\du}}
\pgfusepath{fill}
\pgfsetlinewidth{0.100000\du}
\pgfsetdash{}{0pt}
\pgfsetdash{}{0pt}
\pgfsetmiterjoin
\definecolor{dialinecolor}{rgb}{0.000000, 0.000000, 0.000000}
\pgfsetstrokecolor{dialinecolor}
\pgfpathellipse{\pgfpoint{27.946185\du}{13.506620\du}}{\pgfpoint{2.521926\du}{0\du}}{\pgfpoint{0\du}{1.164960\du}}
\pgfusepath{stroke}
% setfont left to latex
\definecolor{dialinecolor}{rgb}{0.000000, 0.000000, 0.000000}
\pgfsetstrokecolor{dialinecolor}
\node at (27.946185\du,13.701620\du){gersion.o};
\pgfsetlinewidth{0.100000\du}
\pgfsetdash{}{0pt}
\pgfsetdash{}{0pt}
\pgfsetbuttcap
{
\definecolor{dialinecolor}{rgb}{0.000000, 0.000000, 0.000000}
\pgfsetfillcolor{dialinecolor}
% was here!!!
\pgfsetarrowsend{stealth}
\definecolor{dialinecolor}{rgb}{0.000000, 0.000000, 0.000000}
\pgfsetstrokecolor{dialinecolor}
\draw (16.356322\du,3.713882\du)--(19.050000\du,12.132993\du);
}
\pgfsetlinewidth{0.100000\du}
\pgfsetdash{}{0pt}
\pgfsetdash{}{0pt}
\pgfsetbuttcap
{
\definecolor{dialinecolor}{rgb}{0.000000, 0.000000, 1.000000}
\pgfsetfillcolor{dialinecolor}
% was here!!!
\pgfsetarrowsend{stealth}
\definecolor{dialinecolor}{rgb}{0.000000, 0.000000, 1.000000}
\pgfsetstrokecolor{dialinecolor}
\draw (21.350000\du,3.758960\du)--(19.050000\du,12.132993\du);
}
\pgfsetlinewidth{0.100000\du}
\pgfsetdash{}{0pt}
\pgfsetdash{}{0pt}
\pgfsetbuttcap
{
\definecolor{dialinecolor}{rgb}{0.000000, 0.000000, 0.000000}
\pgfsetfillcolor{dialinecolor}
% was here!!!
\pgfsetarrowsend{stealth}
\definecolor{dialinecolor}{rgb}{0.000000, 0.000000, 0.000000}
\pgfsetstrokecolor{dialinecolor}
\draw (46.650000\du,3.869287\du)--(49.800000\du,12.272113\du);
}
\pgfsetlinewidth{0.100000\du}
\pgfsetdash{}{0pt}
\pgfsetdash{}{0pt}
\pgfsetbuttcap
{
\definecolor{dialinecolor}{rgb}{0.000000, 0.000000, 0.000000}
\pgfsetfillcolor{dialinecolor}
% was here!!!
\pgfsetarrowsend{stealth}
\definecolor{dialinecolor}{rgb}{0.000000, 0.000000, 0.000000}
\pgfsetstrokecolor{dialinecolor}
\draw (52.550000\du,3.854206\du)--(49.800000\du,12.272113\du);
}
\pgfsetlinewidth{0.100000\du}
\pgfsetdash{}{0pt}
\pgfsetdash{}{0pt}
\pgfsetbuttcap
{
\definecolor{dialinecolor}{rgb}{0.000000, 0.000000, 0.000000}
\pgfsetfillcolor{dialinecolor}
% was here!!!
\pgfsetarrowsend{stealth}
\definecolor{dialinecolor}{rgb}{0.000000, 0.000000, 0.000000}
\pgfsetstrokecolor{dialinecolor}
\draw (4.096143\du,3.920442\du)--(5.800000\du,12.380635\du);
}
\pgfsetlinewidth{0.100000\du}
\pgfsetdash{}{0pt}
\pgfsetdash{}{0pt}
\pgfsetbuttcap
{
\definecolor{dialinecolor}{rgb}{0.000000, 0.000000, 0.000000}
\pgfsetfillcolor{dialinecolor}
% was here!!!
\pgfsetarrowsend{stealth}
\definecolor{dialinecolor}{rgb}{0.000000, 0.000000, 0.000000}
\pgfsetstrokecolor{dialinecolor}
\draw (9.208936\du,3.969610\du)--(5.800000\du,12.380635\du);
}
\pgfsetlinewidth{0.100000\du}
\pgfsetdash{}{0pt}
\pgfsetdash{}{0pt}
\pgfsetbuttcap
{
\definecolor{dialinecolor}{rgb}{0.000000, 0.000000, 0.000000}
\pgfsetfillcolor{dialinecolor}
% was here!!!
\pgfsetarrowsend{stealth}
\definecolor{dialinecolor}{rgb}{0.000000, 0.000000, 0.000000}
\pgfsetstrokecolor{dialinecolor}
\draw (34.169624\du,3.988717\du)--(37.900000\du,12.307728\du);
}
\pgfsetlinewidth{0.100000\du}
\pgfsetdash{}{0pt}
\pgfsetdash{}{0pt}
\pgfsetbuttcap
{
\definecolor{dialinecolor}{rgb}{0.000000, 0.000000, 0.000000}
\pgfsetfillcolor{dialinecolor}
% was here!!!
\pgfsetarrowsend{stealth}
\definecolor{dialinecolor}{rgb}{0.000000, 0.000000, 0.000000}
\pgfsetstrokecolor{dialinecolor}
\draw (39.650000\du,3.910070\du)--(37.900000\du,12.307728\du);
}
\pgfsetlinewidth{0.100000\du}
\pgfsetdash{}{0pt}
\pgfsetdash{}{0pt}
\pgfsetbuttcap
{
\definecolor{dialinecolor}{rgb}{0.000000, 0.000000, 1.000000}
\pgfsetfillcolor{dialinecolor}
% was here!!!
\pgfsetarrowsend{stealth}
\definecolor{dialinecolor}{rgb}{0.000000, 0.000000, 1.000000}
\pgfsetstrokecolor{dialinecolor}
\draw (22.245003\du,3.669946\du)--(27.268857\du,12.337974\du);
}
\pgfsetlinewidth{0.100000\du}
\pgfsetdash{}{0pt}
\pgfsetdash{}{0pt}
\pgfsetbuttcap
{
\definecolor{dialinecolor}{rgb}{0.000000, 0.000000, 0.000000}
\pgfsetfillcolor{dialinecolor}
% was here!!!
\pgfsetarrowsend{stealth}
\definecolor{dialinecolor}{rgb}{0.000000, 0.000000, 0.000000}
\pgfsetstrokecolor{dialinecolor}
\draw (11.676752\du,3.648435\du)--(26.374288\du,12.554155\du);
}
\pgfsetlinewidth{0.100000\du}
\pgfsetdash{}{0pt}
\pgfsetdash{}{0pt}
\pgfsetbuttcap
{
\definecolor{dialinecolor}{rgb}{0.000000, 0.000000, 0.000000}
\pgfsetfillcolor{dialinecolor}
% was here!!!
\pgfsetarrowsend{stealth}
\definecolor{dialinecolor}{rgb}{0.000000, 0.000000, 0.000000}
\pgfsetstrokecolor{dialinecolor}
\draw (27.682685\du,3.439285\du)--(27.914658\du,12.302076\du);
}
\pgfsetlinewidth{0.100000\du}
\pgfsetdash{}{0pt}
\pgfsetdash{}{0pt}
\pgfsetbuttcap
{
\definecolor{dialinecolor}{rgb}{0.000000, 0.000000, 0.000000}
\pgfsetfillcolor{dialinecolor}
% was here!!!
\pgfsetarrowsend{stealth}
\definecolor{dialinecolor}{rgb}{0.000000, 0.000000, 0.000000}
\pgfsetstrokecolor{dialinecolor}
\draw (38.345608\du,3.784568\du)--(29.094851\du,12.432773\du);
}
\pgfsetlinewidth{0.100000\du}
\pgfsetdash{}{0pt}
\pgfsetdash{}{0pt}
\pgfsetbuttcap
{
\definecolor{dialinecolor}{rgb}{0.000000, 0.000000, 0.000000}
\pgfsetfillcolor{dialinecolor}
% was here!!!
\pgfsetarrowsend{stealth}
\definecolor{dialinecolor}{rgb}{0.000000, 0.000000, 0.000000}
\pgfsetstrokecolor{dialinecolor}
\draw (50.476156\du,3.420838\du)--(29.802659\du,12.675549\du);
}
\pgfsetlinewidth{0.100000\du}
\pgfsetdash{}{0pt}
\pgfsetdash{}{0pt}
\pgfsetbuttcap
{
\definecolor{dialinecolor}{rgb}{0.000000, 0.000000, 1.000000}
\pgfsetfillcolor{dialinecolor}
% was here!!!
\pgfsetarrowsend{stealth}
\definecolor{dialinecolor}{rgb}{0.000000, 0.000000, 1.000000}
\pgfsetstrokecolor{dialinecolor}
\draw (19.973810\du,3.577290\du)--(7.406346\du,12.597106\du);
}
\pgfsetlinewidth{0.100000\du}
\pgfsetdash{}{0pt}
\pgfsetdash{}{0pt}
\pgfsetbuttcap
{
\definecolor{dialinecolor}{rgb}{0.000000, 0.000000, 1.000000}
\pgfsetfillcolor{dialinecolor}
% was here!!!
\pgfsetarrowsend{stealth}
\definecolor{dialinecolor}{rgb}{0.000000, 0.000000, 1.000000}
\pgfsetstrokecolor{dialinecolor}
\draw (22.792721\du,3.553757\du)--(36.250455\du,12.547604\du);
}
\pgfsetlinewidth{0.100000\du}
\pgfsetdash{}{0pt}
\pgfsetdash{}{0pt}
\pgfsetbuttcap
{
\definecolor{dialinecolor}{rgb}{0.000000, 0.000000, 1.000000}
\pgfsetfillcolor{dialinecolor}
% was here!!!
\pgfsetarrowsend{stealth}
\definecolor{dialinecolor}{rgb}{0.000000, 0.000000, 1.000000}
\pgfsetstrokecolor{dialinecolor}
\draw (23.239692\du,3.327553\du)--(47.503545\du,12.803178\du);
}
\definecolor{dialinecolor}{rgb}{1.000000, 1.000000, 1.000000}
\pgfsetfillcolor{dialinecolor}
\pgfpathellipse{\pgfpoint{28.000000\du}{22.912500\du}}{\pgfpoint{2.217202\du}{0\du}}{\pgfpoint{0\du}{1.108601\du}}
\pgfusepath{fill}
\pgfsetlinewidth{0.100000\du}
\pgfsetdash{}{0pt}
\pgfsetdash{}{0pt}
\pgfsetmiterjoin
\definecolor{dialinecolor}{rgb}{0.000000, 0.000000, 0.000000}
\pgfsetstrokecolor{dialinecolor}
\pgfpathellipse{\pgfpoint{28.000000\du}{22.912500\du}}{\pgfpoint{2.217202\du}{0\du}}{\pgfpoint{0\du}{1.108601\du}}
\pgfusepath{stroke}
% setfont left to latex
\definecolor{dialinecolor}{rgb}{0.000000, 0.000000, 0.000000}
\pgfsetstrokecolor{dialinecolor}
\node at (28.000000\du,23.107500\du){gestion};
\pgfsetlinewidth{0.100000\du}
\pgfsetdash{}{0pt}
\pgfsetdash{}{0pt}
\pgfsetbuttcap
{
\definecolor{dialinecolor}{rgb}{0.000000, 0.000000, 0.000000}
\pgfsetfillcolor{dialinecolor}
% was here!!!
\pgfsetarrowsend{stealth}
\definecolor{dialinecolor}{rgb}{0.000000, 0.000000, 0.000000}
\pgfsetstrokecolor{dialinecolor}
\draw (7.957806\du,14.640581\du)--(26.243607\du,22.187592\du);
}
\pgfsetlinewidth{0.100000\du}
\pgfsetdash{}{0pt}
\pgfsetdash{}{0pt}
\pgfsetbuttcap
{
\definecolor{dialinecolor}{rgb}{0.000000, 0.000000, 0.000000}
\pgfsetfillcolor{dialinecolor}
% was here!!!
\pgfsetarrowsend{stealth}
\definecolor{dialinecolor}{rgb}{0.000000, 0.000000, 0.000000}
\pgfsetstrokecolor{dialinecolor}
\draw (20.068784\du,14.394196\du)--(27.034204\du,21.875214\du);
}
\pgfsetlinewidth{0.100000\du}
\pgfsetdash{}{0pt}
\pgfsetdash{}{0pt}
\pgfsetbuttcap
{
\definecolor{dialinecolor}{rgb}{0.000000, 0.000000, 0.000000}
\pgfsetfillcolor{dialinecolor}
% was here!!!
\pgfsetarrowsend{stealth}
\definecolor{dialinecolor}{rgb}{0.000000, 0.000000, 0.000000}
\pgfsetstrokecolor{dialinecolor}
\draw (27.953139\du,14.721967\du)--(27.993370\du,21.753701\du);
}
\pgfsetlinewidth{0.100000\du}
\pgfsetdash{}{0pt}
\pgfsetdash{}{0pt}
\pgfsetbuttcap
{
\definecolor{dialinecolor}{rgb}{0.000000, 0.000000, 0.000000}
\pgfsetfillcolor{dialinecolor}
% was here!!!
\pgfsetarrowsend{stealth}
\definecolor{dialinecolor}{rgb}{0.000000, 0.000000, 0.000000}
\pgfsetstrokecolor{dialinecolor}
\draw (36.598148\du,14.868021\du)--(29.081000\du,21.901110\du);
}
\pgfsetlinewidth{0.100000\du}
\pgfsetdash{}{0pt}
\pgfsetdash{}{0pt}
\pgfsetbuttcap
{
\definecolor{dialinecolor}{rgb}{0.000000, 0.000000, 0.000000}
\pgfsetfillcolor{dialinecolor}
% was here!!!
\pgfsetarrowsend{stealth}
\definecolor{dialinecolor}{rgb}{0.000000, 0.000000, 0.000000}
\pgfsetstrokecolor{dialinecolor}
\draw (47.573962\du,14.640705\du)--(29.739383\du,22.177451\du);
}
\end{tikzpicture}


\item Lorsqu'on exécute \texttt{getsion}, on s'aperçoit que le fichier
  de sortie n'est pas créé. En observant le fichier
  \texttt{clients\_out.c}, on voit que les fonctions
  \texttt{ecrire\_les\_clients} et \texttt{ecrire\_les\_elements} sont
  à compléter. Il est donc normal que le fichier de sortie ne soit pas
  créé.

\item Exemple d'implémentation de la fonction \texttt{ecrire\_les\_clients}:
  \lstinputlisting{listing/clients_out.c}

\item Lorqu'un client se fait un virement à lui-meme, son compte reste
  identique.

\item Même si un virement peut se décomposer en un retrait et un
  dépot, la fonction \texttt{virement} ne peut pas se contenter
  d'appeler successivement \texttt{retrait} et \texttt{depot} car ces
  deux fonctions demandent à chaque fois à l'utilisateur de saisir le
  montant de l'opération. L'utilisateur pourrait alors choisir deux
  montants différents et faire ainsi un bénéfice substantiel.

\item Dans la fonction \texttt{ecrire\_les\_elements}, le nombre
  total d'éléments à écrire s'obtient en parcourant le tableau
  \texttt{ens} et en sommant les case à 1 entre les indices 0 et
  \texttt{e->nb}.

\item Dans le fichier source \texttt{sous\_ensembles.c}, on utilise les
  fichiers d'interface suivants :
  \begin{itemize}
  \item \texttt{\#include "clients.h"}
  \item \texttt{\#include "clients\_out.h"}
  \item \texttt{\#include "clients\_in.h"}
  \end{itemize}

\item Dans le fichier \texttt{sous\_ensembles.c}, on peut par exemple
  implémenter les filtres suivants :
  \lstinputlisting{listing/sous_ensembles.c}



\end{enumerate}

\end{document}
