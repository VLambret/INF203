\documentclass[10pt]{article}

\usepackage{graphics}
\usepackage{dirtree}
\usepackage{paracol}
\usepackage{epigraph}
\usepackage{enumitem}
\usepackage{xcolor}

\usepackage{pgf}
\usepackage{tikz}
\usetikzlibrary{arrows,automata}

\setlength{\textwidth}{17cm}
\setlength{\oddsidemargin}{-1cm}
\setlength{\evensidemargin}{-1cm}
\setlength{\textheight}{26cm}
\setlength{\parindent}{0pt}
\setlength{\parskip}{0.3ex}

\usepackage{fancyhdr}
\usepackage{epsfig}
\usepackage[utf8]{inputenc}
\usepackage[french]{babel}
\usepackage[T1]{fontenc}
%\usepackage{verbatim}
\usepackage{graphics}
\usepackage{amsmath,amsfonts,amssymb}
\usepackage{listings}
\usepackage{thmbox}
\usepackage{comment}

\oddsidemargin=0cm
\evensidemargin=0cm
\textwidth=17cm
\textheight=25cm
\topmargin=0mm
\hoffset=-6mm
\voffset=-25mm
%\headsep=30pt

\newcounter{numpartie}
\newcounter{exo}
\newcommand{\partie}[2]
{
        \refstepcounter{numpartie}
        \setcounter{exo}{0}
        \vspace{3mm}

        %\noindent{\large\bf Partie \Alph{numpartie}~-~#1}\hfill
        \noindent{\large\bf Partie \Alph{numpartie}~ ~#1}\hfill
        %[\;{\it bar\^eme indicatif : #2}\;]\\
        \\

}

\newenvironment{exercice}{\refstepcounter{exo} \vspace*{1em}\begin{thmbox}[S]{{\bf Exercice \arabic{exo}~:}}}{\end{thmbox}}

\newenvironment{solution}{\vspace*{1em}\begin{thmbox}[S]{{\bf Solution~:}}}{\end{thmbox}}

\newcommand{\IGNORE}[1]{}

\def\EnteteUE{
\noindent{\bf Université Grenoble Alpes} \hfill {\bf DLST}

\vspace{-8pt}

\noindent\hrulefill

\vspace{-2pt}

\noindent{\bf UE INF203} \hfill {\bf Année 2017-18}

\vspace*{.5cm}
}

\newcommand{\PiedDePageUE}[1]{
\fancypagestyle{monstyle}
{
\rhead{}
\lfoot{{\bf\small INF203 - 2016/2017}}
\cfoot{\thepage}
\rfoot{{\it #1}}
\renewcommand{\footrulewidth}{0.5pt}
\renewcommand{\headrulewidth}{0.0pt}
}
\pagestyle{monstyle}
}


\newcounter{question}
\newenvironment{question}[1][]{\refstepcounter{question}
   \textbf{[\bf \alph{question}] {\begin{emph}{#1}\end{emph}}}}{~$\blacksquare$~\\}

\newcommand{\unix}[1]{\hspace*{2cm}{\bf \tt #1}}
\newcommand{\fich}[1]{{\bf \em #1}}
\newcommand{\ecran}[1]{{\tt #1}}
\newcommand{\C}[1]{{\tt #1}}
\newcommand{\cmd}[1]{{\bf #1}}
\newcommand{\home}{\~{}}

\newenvironment{exocomp}{\vspace*{1em}\begin{thmbox}[M]{Exercice complémentaire :~}}{\end{thmbox}}


\begin{document}
\thispagestyle{empty}

\definecolor{mygreen}{rgb}{0,0.6,0}
\definecolor{mygray}{rgb}{0.5,0.5,0.5}
\definecolor{mymauve}{rgb}{0.58,0,0.82}

\lstdefinestyle{customc}{
  belowcaptionskip=1\baselineskip,
  breaklines=true,
  frame=L,
  xleftmargin=\parindent,
  language=C,
  showstringspaces=false,
  basicstyle=\footnotesize\ttfamily,
  keywordstyle=\bfseries\color{green!40!black},
  commentstyle=\itshape\color{purple!40!black},
  identifierstyle=\color{blue},
  stringstyle=\color{orange},
}

\lstdefinestyle{none}{
  belowcaptionskip=1\baselineskip,
  breaklines=true,
  frame=L,
  xleftmargin=\parindent,
  language=bash,
  showstringspaces=false,
  basicstyle=\footnotesize\ttfamily,
  keywordstyle=\bfseries\color{black},
  commentstyle=\itshape\color{black},
  identifierstyle=\color{black},
  stringstyle=\color{black},
}

\lstset{tabsize=3, style=customc,literate=
  {á}{{\'a}}1 {é}{{\'e}}1 {í}{{\'i}}1 {ó}{{\'o}}1 {ú}{{\'u}}1
  {Á}{{\'A}}1 {É}{{\'E}}1 {Í}{{\'I}}1 {Ó}{{\'O}}1 {Ú}{{\'U}}1
  {à}{{\`a}}1 {è}{{\`e}}1 {ì}{{\`i}}1 {ò}{{\`o}}1 {ù}{{\`u}}1
  {À}{{\`A}}1 {È}{{\'E}}1 {Ì}{{\`I}}1 {Ò}{{\`O}}1 {Ù}{{\`U}}1
  {ä}{{\"a}}1 {ë}{{\"e}}1 {ï}{{\"i}}1 {ö}{{\"o}}1 {ü}{{\"u}}1
  {Ä}{{\"A}}1 {Ë}{{\"E}}1 {Ï}{{\"I}}1 {Ö}{{\"O}}1 {Ü}{{\"U}}1
  {â}{{\^a}}1 {ê}{{\^e}}1 {î}{{\^i}}1 {ô}{{\^o}}1 {û}{{\^u}}1
  {Â}{{\^A}}1 {Ê}{{\^E}}1 {Î}{{\^I}}1 {Ô}{{\^O}}1 {Û}{{\^U}}1
  {œ}{{\oe}}1 {Œ}{{\OE}}1 {æ}{{\ae}}1 {Æ}{{\AE}}1 {ß}{{\ss}}1
  {ű}{{\H{u}}}1 {Ű}{{\H{U}}}1 {ő}{{\H{o}}}1 {Ő}{{\H{O}}}1
  {ç}{{\c c}}1 {Ç}{{\c C}}1 {ø}{{\o}}1 {å}{{\r a}}1 {Å}{{\r A}}1
  {€}{{\euro}}1 {£}{{\pounds}}1 {«}{{\guillemotleft}}1
  {»}{{\guillemotright}}1 {ñ}{{\~n}}1 {Ñ}{{\~N}}1 {¿}{{?`}}1
}

\EnteteUE

\begin{center}
  {\large {\bf Corrigé TP8}}
\end{center}

\vspace*{0.5cm}

\begin{enumerate}[label=\textbf{[\alph*]}]
  \setlength\itemsep{1em}

\item Le code de la porte est 1234

\item L'arrêt de la simulation est commandé par l'entrée ``\texttt{-}''

\item Oui

\item On stocke la seule sortie au milieu d'un immense tableau qui
  autorise le stockage d'autant de sorties qu'il y a de transitions. Voir
  en annexe le code la version 3 d'automate.c.

  À chaque transition d'état, on affiche la sortie correspondante.

\item On constate que l'implémentation n'autorise pas le changement
  facile du code de la porte. Pour cela, il faut modifier le tableau
  des transitions et le tableau des sorties.

  On peut configurer un code comportant plusieurs fois le même
  chiffre.

\item La version 1 est catastrophique car le code de gestion de
  l'automate et l'automate lui-même sont mélangés dans une seule
  fonction. Il est souvent bon, en programmation, d'isoler les
  concepts afin de pouvoir les modifier séparément les uns des
  autres. On n'a pas envie de modifier le code des transitions
  lorsqu'on souhaite simplement changer le message d'une sortie, par
  exemple. \\

  La version 2 propose une amélioration basique consistant à isoler le
  calcul des transitions dans une fonction. On gagne déjà en
  lisibilité, mais le fonctionnement reste identique. \\

  La version 3 apporte la vraie amélioration : la
  tabulation. L'automate est complètement défini par un tableau. Le
  code ne sert qu'à l'interpréter pour reproduire le comportement
  attendu. Cela va notamment faciliter la sauvegarde et le chargement
  de l'automate dans un fichier externe.

\item Le format retenu pour le fichier contenant le code de la porte
  est de quatre chiffres séparés par une espace.

\begin{verbatim}
1 2 3 4
\end{verbatim}

  Une implémentation possible de la fonction de chargement de
  l'automate Version 4 à partir d'un tel fichier :

  \lstinputlisting[style=customc]{listing/automate4.c}

\item Voir en annexe pour le code complet du programme supprimant les
  commentaires en C.

\item Voir en annexe pour le code complet du programme supprimant les
  commentaires en Shell.

\end{enumerate}

\newpage
\section*{Annexe 1 - Automates}
  Implémentation possible de automate.c Version 1. Moralité : c'est
  super long et pas très clair, vu que tout est mélangé.

  \lstinputlisting[style=customc]{listing/automate1.c}

  \vspace{1cm}
  Implémentation possible de automate.c Version 2.

  \lstinputlisting[style=customc]{listing/automate2.c}

  \vspace{1cm}
  Implémentation possible de automate.c Version 3.

  \lstinputlisting[style=customc]{listing/automate3.c}

  \newpage
  \section*{Annexe 2 - Suppression commentaires C}
  Le code du programme est réparti en 3 fichiers :
  \texttt{automate.c}, \texttt{automate.h},
  \texttt{supp\_comm\_C.c}. Voici le contenu de ces fichiers :

  \lstinputlisting[style=customc,caption=supp\_comm\_sh.c]{listing/commc/supp_comm_C.c}
  \lstinputlisting[style=customc,caption=automate.h]{listing/commc/automate.h}

  \newpage
  \lstinputlisting[style=customc,caption=automate.c]{listing/commc/automate.c}

  \newpage
  \section*{Annexe 3 - Suppression commentaires Shell}
  On reprend largement le code de l'automate précédent pour supprimer
  les commentaires en Shell.

  \lstinputlisting[style=customc,caption=supp\_comm\_sh.c]{listing/commsh/supp_comm_sh.c}
  \lstinputlisting[style=customc,caption=automate.h]{listing/commsh/automate.h}

  \newpage
  \lstinputlisting[style=customc,caption=automate.c]{listing/commsh/automate.c}

\end{document}
